\section{Code Implementation}

To validate our theoretical framework, we have implemented a simulation environment in Python that allows us to test both our mathematical models and their real-world applicability. The codebase consists of three main modules: diagonalization, bit error rate (BER) analysis, and a heatmap generator to visualize the spatial distribution of signal quality.

\subsection{Diagonalization Module}

The diagonalization module implements the core mathematical concepts of our RIS reflection framework, following the theoretical foundation presented in Section 3 and 4.

\subsubsection{Null Space Calculation}
We first start by making the calculation of reflection matrices, which ensure the effective channel between the transmitter and receiver is diagonalized. We implement this through the \texttt{calculate\_W\_single} and \texttt{calculate\_W\_multiple} functions:

\begin{lstlisting}[language=python, caption={Calculation of W matrices}]
def calculate_W_single(K: int, N: int, G: np.ndarray, H: np.ndarray) -> np.ndarray:
    """
    Calculate W matrix for a single receiver.
    
    Args:
        K: Number of antennas
        N: Number of reflecting elements
        G: Channel matrix from RIS to receiver (KxN)
        H: Channel matrix from transmitter to RIS (NxK)
    
    Returns:
        W: The W matrix as defined in equation (8) of the paper
    """
    W = np.zeros((N, N), dtype=complex)
    
    for i in range(K):
        for j in range(K):
            if i != j:
                temp = np.multiply(G[j, :], H[:, i].T)
                W += np.outer(temp.conj(), temp)
    
    return W
\end{lstlisting}

\newpage
This function calculates the W matrix for a single receiver. For multiple receivers, we stack these matrices:

\begin{lstlisting}[language=python, caption={W matrix for multiple receivers}]
def calculate_W_multiple(K: int, N: int, J: int, Gs: List[np.ndarray], H: np.ndarray) -> np.ndarray:
    """
    Calculate combined W matrix for multiple receivers.
    """
    W_combined = np.zeros((J * N, N), dtype=complex)
    
    for j in range(J):
        W_j = calculate_W_single(K, N, Gs[j], H)
        W_combined[j*N:(j+1)*N, :] = W_j
    
    return W_combined
\end{lstlisting}

\subsubsection{RIS Reflection Matrix Calculation}

With the W matrix defined, we can calculate the reflection matrices using Singular Value Decomposition (SVD) to find the null space:

\begin{lstlisting}[language=python, caption={RIS Reflection Matrix Calculation}]
def calculate_ris_reflection_matrice(
    K: int, 
    N: int, 
    J: int, 
    Gs: List[np.ndarray], 
    H: np.ndarray, 
    eta: float,
) -> Tuple[np.ndarray, float]:
    """
    Calculate reflection matrices for RIS surfaces.
    """
    W = calculate_W_multiple(K, N, J, Gs, H)
    U, sigma, Vh = np.linalg.svd(W)
    
    null_space_dim = N - J*K**2 + J*K
    if null_space_dim <= 0:
        raise ValueError(f"No solution exists. Need more reflecting elements.")
    
    first_singular_values = sigma[:N - null_space_dim]
    last_singular_values = sigma[-null_space_dim:]
    
    null_space_basis = Vh[-null_space_dim:, :].T.conj()

    if null_space_basis.shape != (N, null_space_dim):
        raise ValueError(f"Invalid null space basis shape.")

    a = np.random.normal(0, 1, (null_space_dim,)) + 1j * np.random.normal(0, 1, (null_space_dim,))
    
    p_unnormalized = null_space_basis @ a
    p = eta * p_unnormalized / np.max(np.abs(p_unnormalized))
    P = np.diag(p)
    dor = 2 * null_space_dim
    return P, dor
\end{lstlisting}

This function implements the theoretical approach described in Section 4.1, where we find the null space of W and then generate a random vector in this space to ensure randomness in our reflection coefficients.

\subsubsection{Multiple RIS Support}

To support multiple RIS surfaces in series, we implemented:

\begin{lstlisting}[language=python, caption={Multiple RIS calculation}]
def calculate_multi_ris_reflection_matrices(
    K: int, 
    N: int, 
    J: int, 
    M: int,
    Gs: List[np.ndarray], 
    H: np.ndarray, 
    eta: float,
    Cs: List[np.ndarray]
) -> Tuple[List[np.ndarray], float]:
    """
    Calculate reflection matrices for M RIS surfaces.
    """
    if len(Cs) != M-1:
        raise ValueError(f"Expected {M-1} inter-RIS channel matrices.")

    ps = []
    S = np.eye(N)

    # Generate M-1 random reflection vectors
    for i in range(M-1):
        absorptions = np.random.uniform(0, 1, N)
        phases = np.random.uniform(0, 2*np.pi, N)
        # p_m[i] = eta * r_i * exp(j*theta_i)
        p_m = absorptions * np.exp(1j * phases)
        ps.append(p_m)
        S = S @ np.diag(p_m) @ Cs[i]

    Gs_prime = [G @ S for G in Gs]
    
    # Calculate the last reflection matrix
    P_final, dor = calculate_ris_reflection_matrice(K, N, J, Gs_prime, H, eta)
    p_final = np.diag(P_final)
    ps.append(p_final)

    Ps = []
    for pm in ps:
        Ps.append(np.diag(pm))
    
    return Ps, dor
\end{lstlisting}

This implements our approach from Section 4.3, where we generate random reflection coefficients for all but the last RIS, then calculate the last one to ensure diagonalization.

\subsubsection{Unified Reflection Matrix}

For convenience in calculations, we provide a function to unify multiple reflection matrices into a single effective matrix:

\begin{lstlisting}[language=python, caption={Unifying RIS matrices}]
def unify_ris_reflection_matrices(Ps: List[np.ndarray], Cs: List[np.ndarray]) -> np.ndarray:
    """
    Unify reflection matrices into a single matrix.
    """
    P = Ps[0]
    for i in range(len(Ps)-1):
        P = P @ Cs[i] @ Ps[i+1]
    return P
\end{lstlisting}

\subsubsection{Verification}

We also implement verification functions to ensure our theoretical expectations match the implementation:

\begin{lstlisting}[language=python, caption={Verification of diagonalization}]
def verify_multi_ris_diagonalization(
    Ps: List[np.ndarray],
    Gs: List[np.ndarray],
    H: np.ndarray,
    Cs: List[np.ndarray]
) -> List[bool]:
    """
    Verify that G(P_1C_1P_2C_2...P_M)H is diagonal for all receivers.
    """
    results = []
    
    P = unify_ris_reflection_matrices(Ps, Cs)
    for G in Gs:
        effective_channel = G @ P @ H
        off_diag_sum = np.sum(np.abs(effective_channel - np.diag(np.diag(effective_channel))))
        results.append(off_diag_sum < tolerance)
    return results
\end{lstlisting}

\subsection{BER Module}

The BER (Bit Error Rate) module implements the simulation of space shift keying (SSK) transmissions and calculates the error rates under different scenarios.

\subsubsection{SSK Transmission Simulation}

We simulate SSK transmission with the following core functions:

\begin{lstlisting}[language=python, caption={SSK Transmission Simulation}]
def simulate_ssk_transmission(K: int, sigma_sq: float, calculate_detected_id: Callable[[np.ndarray, np.ndarray], float]):
    n_bits = int(np.log2(K))
    if 2**n_bits != K:
        raise ValueError(f"K must be a power of 2, got {K}")
    bit_mappings = np.array([format(i, f'0{n_bits}b') for i in range(K)])    
    true_bits = np.random.randint(0, 2, n_bits)
    true_bits_str = ''.join(map(str, true_bits))
    true_idx = np.where(bit_mappings == true_bits_str)[0][0]

    x = np.zeros(K)
    x[true_idx] = 1

    noise = create_random_noise_vector(K, sigma_sq)
    detected_idx = calculate_detected_id(x, noise)
    
    detected_bits = np.array(list(bit_mappings[detected_idx])).astype(int)
    errors = np.sum(detected_bits != true_bits)
    return errors / n_bits
\end{lstlisting}

\newpage
This function implements the common core of our SSK transmission simulation. It maps bits to antenna indices, simulates transmission with noise, and calculates the error rate.

We then implement specialized versions for reflection and direct transmission:

\begin{lstlisting}[language=python, caption={SSK Transmission with Reflection}]
def simulate_ssk_transmission_reflection(K: int, effective_channel: np.ndarray, sigma_sq: float):
    if effective_channel.shape != (K, K):
        raise ValueError(f"Reflection: Effective channel shape must be ({K}, {K})")

    def calculate_detected_id(x: np.ndarray, noise: np.ndarray):
        y = effective_channel @ x + noise
        return np.argmax(np.abs(y)**2)
    
    return simulate_ssk_transmission(K, sigma_sq, calculate_detected_id)
\end{lstlisting}

\begin{lstlisting}[language=python, caption={SSK Transmission with Direct Path}]
def simulate_ssk_transmission_direct(K: int, B: np.ndarray, effective_channel: np.ndarray, sigma_sq: float):
    if B.shape != (K, K):
        raise ValueError(f"Direct: B shape must be ({K}, {K})")
    
    if effective_channel.shape != (K, K):
        raise ValueError(f"Direct: Effective channel shape must be ({K}, {K})")

    def calculate_detected_id(x: np.ndarray, noise: np.ndarray):
        y = (B + effective_channel) @ x + noise
        distances = np.array([np.linalg.norm(y - B[:, i]) for i in range(B.shape[1])])
        return np.argmin(distances)
    
    return simulate_ssk_transmission(K, sigma_sq, calculate_detected_id)
\end{lstlisting}

These functions implement the detection methods described in Section 3.2, where the legitimate receiver can detect the signal through the diagonalized channel, while the eavesdropper hears from the direct path and receives interference from the reflection.

\subsubsection{BER Simulation}

We also implement a comprehensive BER simulation function that evaluates performance across different SNR values:

\begin{lstlisting}[language=python, caption={BER Simulation}]
def calculate_ber_simulation(snr_db, K, N, J, M, eta=0.9, num_symbols=10000):
    sigma_sq = snr_db_to_sigma_sq(snr_db)
    errors_receiver = 0
    errors_eavesdropper = 0
    errors_direct = 0
    
    errors_receiver_double = 0
    errors_eavesdropper_double = 0
    
    for _ in range(num_symbols):
        # Generate channel matrices
        H = generate_random_channel_matrix(N, K)
        Gs = [generate_random_channel_matrix(K, N) for _ in range(J)]
        G = random.choice(Gs)
        Fs = [generate_random_channel_matrix(K, N) for _ in range(M)]
        B = generate_random_channel_matrix(K, K)
        Cs = [generate_random_channel_matrix(N, N) for _ in range(M-1)]
        
        # Calculate reflection matrices
        Ps, _ = calculate_multi_ris_reflection_matrices(K, N, J, M, Gs, H, eta, Cs)
        P = unify_ris_reflection_matrices(Ps, Cs)

        # Calculate effective channels
        effective_channel_receiver = G @ P @ H
        effective_channel_eavesdropper = np.zeros((K, K), dtype=np.complex128)
        for i in range(M):
            P_to_i = unify_ris_reflection_matrices(Ps[:i+1], Cs[:i])
            effective_channel_eavesdropper += Fs[i] @ P_to_i @ H
        effective_channel_direct = np.zeros((K, K))

        # Simulate transmissions
        errors_receiver += simulate_ssk_transmission_reflection(K, effective_channel_receiver, sigma_sq)
        errors_eavesdropper += simulate_ssk_transmission_direct(K, B, effective_channel_eavesdropper, sigma_sq)
        errors_direct += simulate_ssk_transmission_direct(K, B, effective_channel_direct, sigma_sq)
        
        H2 = generate_random_channel_matrix(N, K)
        Gs2 = [generate_random_channel_matrix(K, N) for _ in range(J)]
        G2 = random.choice(Gs2)
        Fs2 = [generate_random_channel_matrix(K, N) for _ in range(M)]
        Cs2 = [generate_random_channel_matrix(N, N) for _ in range(M-1)]
        Ps2, _ = calculate_multi_ris_reflection_matrices(
            K, N, J, M, Gs2, H2, eta, Cs2
        )
        P2 = unify_ris_reflection_matrices(Ps2, Cs2)

        effective_channel_receiver_2 = G2 @ P2 @ H2
        effective_channel_eavesdropper_2 = np.zeros((K, K), dtype=np.complex128) # F @ P @ H
        for i in range(M):
            P_to_i = unify_ris_reflection_matrices(Ps2[:i+1], Cs2[:i])
            effective_channel_eavesdropper_2 += Fs2[i] @ P_to_i @ H2

        effective_channel_receiver_double = effective_channel_receiver + effective_channel_receiver_2
        effective_channel_eavesdropper_double = effective_channel_eavesdropper + effective_channel_eavesdropper_2

        errors_receiver_double += simulate_ssk_transmission_reflection(K, effective_channel_receiver_double, sigma_sq)
        errors_eavesdropper_double += simulate_ssk_transmission_direct(K, B, effective_channel_eavesdropper_double, sigma_sq)
    
    result_receiver = errors_receiver / num_symbols
    result_eavesdropper = errors_eavesdropper / num_symbols
    result_direct = errors_direct / num_symbols
    result_receiver_double = errors_receiver_double / num_symbols
    result_eavesdropper_double = errors_eavesdropper_double / num_symbols

    return result_receiver, result_eavesdropper, result_direct, result_receiver_double, result_eavesdropper_double
\end{lstlisting}

This function simulates BER performance across legitimate receivers and eavesdroppers, including scenarios with multiple RIS surfaces and different path configurations.

\newpage
\subsection{Heatmap Generator}

To visualize our results in a spatial context, we implemented a heatmap generator that simulates signal quality across a 2D space:

\subsubsection{Core Heatmap Class}

\begin{lstlisting}[language=python, caption={Heatmap Generator Class}]
class HeatmapGenerator:
    def __init__(self, width: int, height: int, resolution: float = 0.5):
        """
        Initialize the heatmap generator with given dimensions.
        """
        self.width = width
        self.height = height
        self.resolution = resolution
        
        # Calculate grid dimensions based on resolution
        self.grid_width = int(width / resolution)
        self.grid_height = int(height / resolution)
        self.grid = np.zeros((self.grid_height, self.grid_width))
        self.buildings = []
        # Dictionary to store points with their labels and coordinates
        self.points = {}
\end{lstlisting}

The heatmap generator creates a grid-based representation of a physical space, where we can place transmitters, receivers, and obstacles.

\subsubsection{Building and Point Management}

\begin{lstlisting}[language=python, caption={Buildings and Points in Heatmap}]
def add_building(self, x: int, y: int, width: int, height: int):
    """
    Add a building to the map. Buildings are excluded from the heatmap calculation.
    """
    self.buildings.append((x, y, width, height))
    grid_x, grid_y = self._meters_to_grid(x, y)
    grid_width = int(width / self.resolution)
    grid_height = int(height / self.resolution)
    
    # Mark building area as NaN to exclude from heatmap
    self.grid[grid_y:grid_y+grid_height, grid_x:grid_x+grid_width] = np.nan

def add_point(self, label: str, x: float, y: float):
    """
    Add a point of interest to the map with a specific label.
    """
    if not (0 <= x < self.width and 0 <= y < self.height):
        raise ValueError(f"Point {label} coordinates ({x}, {y}) are outside the map boundaries")
    self.points[label] = (x, y)
\end{lstlisting}

These functions allow us to define the simulation environment with buildings (which block signals) and points representing transmitters, RIS surfaces, and receivers.

\newpage
\subsubsection{Line of Sight Checking}

An important aspect of our simulation is determining whether two points have line-of-sight:

\begin{lstlisting}[language=python, caption={Line of Sight Checking}]
def _line_intersects_building(self, x1: float, y1: float, x2: float, y2: float) -> bool:
    """
    Check if line between two points intersects any building.
    Uses line segment intersection algorithm.
    """
    def ccw(A: tuple, B: tuple, C: tuple) -> bool:
        """Returns True if points are counter-clockwise oriented"""
        return (C[1] - A[1]) * (B[0] - A[0]) > (B[1] - A[1]) * (C[0] - A[0])

    def intersect(A: tuple, B: tuple, C: tuple, D: tuple) -> bool:
        """Returns True if line segments AB and CD intersect"""
        return ccw(A, C, D) != ccw(B, C, D) and ccw(A, B, C) != ccw(A, B, D)

    for bx, by, bw, bh in self.buildings:
        building_corners = [
            (bx, by), (bx + bw, by),
            (bx + bw, by + bh), (bx, by + bh)
        ]
        
        for i in range(4):
            if intersect(
                (x1, y1), (x2, y2),
                building_corners[i], building_corners[(i + 1) % 4]
            ):
                return True
    return False
\end{lstlisting}

This function allows us to determine if buildings block the line-of-sight between points, which is crucial for accurate path loss simulation.

\subsubsection{Distance Calculation}

To model path loss, we calculate distances between points:

\begin{lstlisting}[language=python, caption={Distance Calculation}]
def calculate_distance_from_point(self, point: str) -> np.ndarray:
    """
    Calculate the minimum distance from each grid cell to the specified point.
    """
    distances = np.full_like(self.grid, np.inf)
    px, py = self.points[point]
    
    for grid_y in range(self.grid_height):
        for grid_x in range(self.grid_width):
            if np.isnan(self.grid[grid_y, grid_x]):
                continue
                
            x, y = self._grid_to_meters(grid_x, grid_y)
            if self._line_intersects_building(x, y, px, py):
                continue
                
            distance = np.sqrt((x - px)**2 + (y - py)**2)
            distances[grid_y, grid_x] = distance
            
    return distances
\end{lstlisting}

This function creates a grid where each cell contains the distance to a specified point, setting infinite distance for points without line-of-sight due to buildings.

\newpage
\subsubsection{Channel Model Functions}

We implement realistic channel models for our simulations:

\begin{lstlisting}[language=python, caption={Channel Modeling Functions}]
def calculate_free_space_path_loss(d: float, lam = 0.08, k = 2) -> float:
    """
    Calculate free space path loss between transmitter and receiver
    """
    if d == 0: d = 0.01
    return 1 / np.sqrt((4 * np.pi / lam) ** 2 * d ** k)

def calculate_unit_spatial_signature(incidence: float, K: int, delta: float):
    """
    Calculate the unit spatial signature vector for a given angle of incidence
    """
    directional_cosine = np.cos(incidence)
    e = np.array([(1 / np.sqrt(K)) * np.exp(-1j * 2 * np.pi * (k - 1) * delta * directional_cosine) for k in range(K)])
    return e.reshape(-1, 1)

def generate_rice_faiding_channel(L: int, K: int, ratio: float, total_power = 1.0) -> np.ndarray:
    """
    Generate a Ricean fading channel matrix
    """
    nu = np.sqrt(ratio * total_power / (1 + ratio))
    sigma = np.sqrt(total_power / (2 * (1 + ratio)))
    return generate_rice_matrix(L, K, nu, sigma)

def calculate_mimo_channel_gain(d: float, L: int, K: int, lam = 0.08, k = 2) -> tuple[np.ndarray, float]:
    """
    Calculate MIMO channel gains between transmitter and receiver
    """
    if d == np.inf:
        return np.zeros((K, L), dtype=complex)
    if d == 0:
        d = 0.5

    delta = lam / 2
    c = np.sqrt(L * K) * np.exp(-1j * 2 * np.pi * d / lam)
    e_r = calculate_unit_spatial_signature(0, K, delta)
    e_t = calculate_unit_spatial_signature(0, L, delta)
    H = c * (e_r @ e_t.T.conj())

    ratio = 0.6
    total_power = 1.0
    H = H * generate_rice_faiding_channel(L, K, ratio, total_power)
    return H
\end{lstlisting}

These functions implement the channel models described in Section 5.2, including free space path loss, spatial signatures, and Ricean fading.

\newpage
\subsubsection{BER Heatmap Simulation}

Finally, we put everything together in a function that simulates BER across the entire space:

\begin{lstlisting}[language=python, caption={BER Heatmap Simulation}]
def ber_heatmap_reflection_simulation(
    width: int,
    height: int,
    buildings: List[Tuple[int, int, int, int]],
    transmitter: Tuple[int, int],
    ris_points: List[Tuple[int, int]],
    receivers: List[Tuple[int, int]],
    num_symbols: int,
    N: int = 16,
    K: int = 2,
    eta: float = 0.9,
    snr_db: int = 10,
    path_loss_calculation_type: Literal['sum', 'product', 'active_ris'] = 'sum'
):
    """
    Run RIS reflection simulation with given parameters
    """
    ber_heatmap = HeatmapGenerator(width, height)
    
    # Setup environment
    for building in buildings:
        ber_heatmap.add_building(*building)

    tx, ty = transmitter
    ber_heatmap.add_point('T', tx, ty)

    M = len(ris_points)
    for i, (px, py) in enumerate(ris_points):
        ber_heatmap.add_point(f'P{i+1}', px, py)

    J = len(receivers)
    for i, (rx, ry) in enumerate(receivers):
        ber_heatmap.add_point(f'R{i+1}', rx, ry)

    # Calculate distance matrices
    distances_from_T = ber_heatmap.calculate_distance_from_point('T')
    distances_from_Ps = [ber_heatmap.calculate_distance_from_point(f'P{i+1}') for i in range(M)]

    # Create heatmaps for power analysis
    power_heatmap_from_T = HeatmapGenerator.copy_from(ber_heatmap)
    power_heatmap_from_Ps = [HeatmapGenerator.copy_from(ber_heatmap) for _ in range(M)]

    # Calculate channel matrices
    tx_grid_y, tx_grid_x = ber_heatmap._meters_to_grid(tx, ty)
    H = calculate_mimo_channel_gain(distances_from_Ps[0][tx_grid_y, tx_grid_x], K, N)

     if M > 1:
        receiver_grid_coords = [(ber_heatmap._meters_to_grid(rx, ry)) for rx, ry in receivers]
        Gs = [calculate_mimo_channel_gain(distances_from_Ps[-1][ry, rx], N, K) 
              for ry, rx in receiver_grid_coords]

        ris_grid_coords = [ber_heatmap._meters_to_grid(px, py) for px, py in ris_points]
        Cs = [calculate_mimo_channel_gain(
            distances_from_Ps[i+1][ris_grid_coords[i][1], ris_grid_coords[i][0]], 
            N, N
        ) for i in range(M-1)]
    else:
        receiver_grid_coords = [(ber_heatmap._meters_to_grid(rx, ry)) for rx, ry in receivers]
        Gs = [calculate_mimo_channel_gain(distances_from_Ps[0][ry, rx], N, K) 
              for ry, rx in receiver_grid_coords]
        Cs = []

    print(f"Channel matrix from transmitter to RIS: Power {calculate_channel_power(H):.1e}")
    print(f"Channel matrix from RIS to receiver: Power {calculate_channel_power(Gs[0]):.1e}")
    
    Ps, _ = calculate_multi_ris_reflection_matrices(K, N, J, M, Gs, H, eta, Cs)
    P = unify_ris_reflection_matrices(Ps, Cs)
    print(f"Reflection matrix: Power {calculate_channel_power(P):.1e}")
    print(f"Effective channel matrix: Power {calculate_channel_power(Gs[0] @ P @ H):.1e}")
    print()

    # * Calculate cumulative path distances
    ris_path_distances = []
    for i in range(M):
        if i == 0:
            # * Distance from T to first RIS
            ris_path_distances.append(distances_from_Ps[0][ty, tx])
        else:
            # * Distance between consecutive RIS points
            ris_path_distances.append(
                distances_from_Ps[i][ris_points[i-1][1], ris_points[i-1][0]]
            )

    # Define BER calculation function for each point
    def calculate_ber_per_point(x: int, y: int) -> float:
        grid_x, grid_y = ber_heatmap._meters_to_grid(x, y)
        distance_from_T = distances_from_T[grid_y, grid_x]
        B = calculate_mimo_channel_gain(distance_from_T, K, K) * calculate_free_space_path_loss(distance_from_T)
        B_power = calculate_channel_power(B)
        power_heatmap_from_T.grid[grid_y, grid_x] = B_power

        distances_from_Ps_current = [distances_from_Ps[i][grid_y, grid_x] for i in range(M)]
        Fs = [calculate_mimo_channel_gain(d, N, K) for d in distances_from_Ps_current]
        
        # * Override channel matrices for receiver positions
        for j in range(J):
            if x == receivers[j][0] and y == receivers[j][1]:
                Fs[-1] = Gs[j]

        errors = 0
        for _ in range(num_symbols):
            Ps, _ = calculate_multi_ris_reflection_matrices(K, N, J, M, Gs, H, eta, Cs)
            P = unify_ris_reflection_matrices(Ps, Cs)

            effective_channel = np.zeros((K, K), dtype=complex)
            
            # Handle different path loss types
            for i in range(M):
                if i == 0:
                    P_to_i = Ps[0]
                else:
                    P_to_i = unify_ris_reflection_matrices(Ps[:i+1], Cs[:i])
                
                if path_loss_calculation_type == 'sum':
                    total_distance = sum(ris_path_distances[:i+1]) + distances_from_Ps_current[i]
                    total_path_loss = calculate_free_space_path_loss(total_distance)
                    new_effective_channel = Fs[i] @ P_to_i @ H * total_path_loss
                elif path_loss_calculation_type == 'product':
                    total_path_loss = 1
                    for j in range(i+1):
                        total_path_loss *= calculate_free_space_path_loss(ris_path_distances[j])
                    total_path_loss *= calculate_free_space_path_loss(distances_from_Ps_current[i])
                    new_effective_channel = Fs[i] @ P_to_i @ H * total_path_loss
                elif path_loss_calculation_type == 'active_ris':
                    total_path_loss = calculate_free_space_path_loss(distances_from_Ps_current[i])
                    new_effective_channel = Fs[i] @ P_to_i @ H * total_path_loss 
                else: 
                    raise ValueError(f"Invalid path loss calculation type: {path_loss_calculation_type}")   
                
                effective_channel += new_effective_channel
            
            # Determine signal power and calculate BER
            power = B_power if distance_from_T != np.inf else calculate_channel_power(effective_channel) 
            sigma_sq = snr_db_to_sigma_sq(snr_db, power)
            
            if distance_from_T == np.inf:
                errors += simulate_ssk_transmission_reflection(K, effective_channel, sigma_sq)
            else:
                errors += simulate_ssk_transmission_direct(K, B, effective_channel, sigma_sq)
                
        return errors / num_symbols

    # Apply BER calculation to each point in the grid
    ber_heatmap.apply_function(calculate_ber_per_point)
    
    # Visualize results
    title = f'{M} RIS(s) (K = {K}, SNR = {snr_db}) [Path Loss: {path_loss_calculation_type}]'
    ber_heatmap.visualize(title + ' BER Heatmap', vmin=0.0, vmax=1.0, label='BER', show_receivers_values=True)
    ber_heatmap.visualize(title + ' BER Heatmap', log_scale=True, vmin=-10.0, vmax=0.0, label='BER', show_receivers_values=True)

    power_heatmap_from_T.visualize(title + ' Channel Power from Transmitter', log_scale=True, vmin=-10.0, vmax=0.0, label='Power (dB)')
    for i in range(M):
        power_heatmap_from_Ps[i].visualize(title + f' Channel Power from RIS {i+1}', log_scale=True, vmin=-10.0, vmax=0.0, label='Power (dB)')
\end{lstlisting}

This function implements our full simulation, calculating BER at each point in the space based on our theoretical framework and the different path loss models discussed in Section 5.2.

\newpage
\subsection{Main Simulation Scenarios}

The main function in the heatmap.py file demonstrates how we use these components to evaluate different scenarios:

\begin{lstlisting}[language=python, caption={Main Simulation Scenarios}]
def main():
    # One reflection simulation
    buildings_single = [
        (0, 10, 7, 10),
        (8, 0, 12, 8)
    ]
    transmitter_single = (3, 3)
    ris_points_single = [(7, 9)]
    receivers_single = [(16, 11), (10, 18)]
    
    for path_loss_calculation_type in PATH_LOSS_TYPES:
        ber_heatmap_reflection_simulation(
            width=20,
            height=20,
            buildings=buildings_single,
            transmitter=transmitter_single,
            ris_points=ris_points_single,
            receivers=receivers_single,
            N=25,
            K=4,
            path_loss_calculation_type=path_loss_calculation_type,
            num_symbols=num_symbols
        )

    # Multiple reflection simulation
    buildings_multiple = [
        (0, 10, 10, 10),
        (2, 4, 7, 1)
    ]
    transmitter_multiple = (1, 1)
    ris_points_multiple = [(0, 9), (10, 9)]
    receivers_multiple = [(16, 14), (12, 18)]
    
    for path_loss_calculation_type in PATH_LOSS_TYPES:
        ber_heatmap_reflection_simulation(
            width=20,
            height=20,
            buildings=buildings_multiple,
            transmitter=transmitter_multiple,
            ris_points=ris_points_multiple,
            receivers=receivers_multiple,
            N=16,
            K=2,
            path_loss_calculation_type=path_loss_calculation_type,
            num_symbols=num_symbols
        )
\end{lstlisting}

This function sets up and runs simulations for two distinct scenarios:
1. A single RIS reflection setup with two buildings, a transmitter, and two receivers
2. A multiple RIS reflection setup with two RIS surfaces in series

Each scenario is simulated with all three path loss models: sum, product, and active RIS. These simulations generate the heatmaps presented in Section 5.2, allowing us to visualize how BER varies across space under different conditions.

The simulations for the BER plots shown in section 5.1 are implemented in the `plot\_ber\_curves()` function:

\begin{lstlisting}[language=python, caption={BER Curve Plotting Function}]
def plot_ber_curves():
    N = 16    # Number of reflecting elements
    K = 2     # Number of antennas 
    eta = 0.9 # Reflection efficiency
    
    for J in range(1, 3):  # Number of receivers
        for M in range(1, 3):  # Number of RIS surfaces
            print(f"Processing J={J}, M={M}")
            snr_range_db = np.arange(-10, 31, 2)
            ber_simulated_receiver = []
            ber_simulated_eavesdropper = []
            ber_simulated_direct = []
            ber_simulated_receiver_double = []
            ber_simulated_eavesdropper_double = []
            
            for snr_db in snr_range_db:
                result_receiver, result_eavesdropper, result_direct, result_receiver_double, result_eavesdropper_double = calculate_ber_simulation(snr_db, K, N, J, M, eta)
                ber_simulated_receiver.append(result_receiver)
                ber_simulated_eavesdropper.append(result_eavesdropper)
                ber_simulated_direct.append(result_direct)
                ber_simulated_receiver_double.append(result_receiver_double)
                ber_simulated_eavesdropper_double.append(result_eavesdropper_double)
                print(f"Processed SNR = {snr_db} dB:\t{result_receiver:.2f}\t{result_eavesdropper:.2f}\t{result_direct:.2f}")

            plt_name = f'SSK BER Performance with RIS (K={K}, N={N}, J={J}, M={M})'
            plt.figure(figsize=(10, 6))
            plt.semilogy(snr_range_db, ber_simulated_direct, label=f'Simulation Direct')
            plt.semilogy(snr_range_db, ber_simulated_receiver, label='Simulation Receiver')
            plt.semilogy(snr_range_db, ber_simulated_receiver_double, label='Simulation Receiver Double RIS Source')
            plt.semilogy(snr_range_db, ber_simulated_eavesdropper, label=f'Simulation Eavesdropper')
            plt.semilogy(snr_range_db, ber_simulated_eavesdropper_double, label=f'Simulation Eavesdropper Double RIS Source')
            plt.grid(True)
            plt.xlabel('SNR (dB)')
            plt.ylabel('Bit Error Rate (BER)')
            plt.title(plt_name)
            plt.legend()
            plt.savefig(f"./simulations/results/{plt_name}.png", dpi=300, format='png')
            print(f"Saved {plt_name}.png\n\n")
\end{lstlisting}

This function systematically evaluates BER performance across different signal-to-noise ratios (SNR), generating plots for each combination of receiver count (J) and RIS surface count (M). These plots allow us to compare the performance of legitimate receivers versus eavesdroppers under different conditions.

\newpage
\subsection{Utils Module}

Although not shown in the provided code snippets, we also implemented a util module that handles noise generation and SNR calculations:

\begin{lstlisting}[language=python, caption={Util Module Functions}]
def snr_db_to_sigma_sq(snr_db: float, power: float = 1.0) -> float:
    """
    Convert SNR in dB to noise variance
    
    Parameters:
    -----------
    snr_db : SNR in dB
    power : Signal power (default 1.0)
    
    Returns:
    --------
    Noise variance sigma_sq
    """
    snr_linear = 10**(snr_db/10)
    return power / snr_linear

def create_random_noise_vector(size: int, sigma_sq: float) -> np.ndarray:
    """
    Create a random noise vector with specified variance
    
    Parameters:
    -----------
    size : Size of the noise vector
    sigma_sq : Noise variance
    
    Returns:
    --------
    Random noise vector
    """
    return np.random.normal(0, np.sqrt(sigma_sq/2), (size,)) + 1j * np.random.normal(0, np.sqrt(sigma_sq/2), (size,))
\end{lstlisting}
